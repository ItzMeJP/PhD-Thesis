%-----------------------------------------------------------------------------------------------------------------
% Resumo em Português
%-----------------------------------------------------------------------------------------------------------------
\begin{otherlanguage}{portuguese}

\begin{center}
\large{Planeamento de preensão robótica adaptável: \\ Uma nova arquitetura \textit{pipeline} de preensão robótica unificada e modular}

\vskip5mm
\normalsize{\textit{João Pedro Carvalho de Souza}}


\vskip5mm
\small{Submetido na Universidade de Trás-os-Montes e Alto Douro \\
para o preenchimento dos requisitos parciais para obtenção do grau de \\
Doutor em Engenharia Electrotécnica e de Computadores}
\end{center}



\textbf{Resumo ---}
\
A preensão robótica persiste como um problema na indústria moderna que busca técnicas autónomas, de implementação rápida e de alta eficiência em cenários complexos. A implantação de uma estrutura de preensão modular e reconfi-gurável para robôs atendendo demandas reais é modesta, mesmo com várias metodo-logias propostas. Cientificamente, a comunidade robótica carece de uma arquitetura de \textit{pipeline} de preensão bem estruturada e formalizada que organize abordagens que permitam a avaliação e a base para novos avanços. Ao oferecer este novo \textit{pipeline} de preensão, dotado de interface de utilizador adequada e metodologias incorporadas, a comunidade científica terá à sua disposição uma estrutura baseada em software para fácil integração e teste de novas contribuições científicas. A indústria terá uma ferramenta intuitiva e poderosa capaz de resolver o problema de preensão em diversos cenários.

\textbf{Palavras Chave:} Arquitetura de Software Modular, Planeamento de Preensão; Manipulação Robótica.

%-----------------------------------------------------------------------------------------------------------------
\end{otherlanguage}
