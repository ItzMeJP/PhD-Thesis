\chapter{Conclusion and Future Work}
\label{cap6:conclusion}

\section{Conclusion}

An important and decisive factor that production lines have to deal with in the Industry 4.0 challenge is flexibility and modularity. In addition to reducing costs and improving efficiency, companies need to be able to respond quickly to modern market demand, which translates into high adaptability of the shop floor. Therefore, robotic and automated production lines have evolved from a requirement to a necessity. In this context, a common type of application is the robotic grasping in assembly or production lines. 

In the academic and scientific field, the robotic grasping procedure is still an unsolved problem and various proposals are constantly made in the current literature to develop a solid and generic approach. However, there is a gap in a well-structured and formalised architecture that organises approaches that allow evaluation and form the basis for new advances in science.

Based on these issues, the present thesis proposes an architecture for a robotic grasping planner in the form of a pipeline that is easily configurable, in a modular fashion, portable and unified.

The proposed pipeline is able to integrate well consolidated software, methods and strategies that focus on grasping, such as the ''GraspIt!" simulator and the \ac{SANN} algorithm, and that do not focus on grasping, such as 6D Mimic and OR, in the form of plugins or not. These are examples that were deployed during the development of the present work and show the ability to integrate new solutions and methods into the pipeline.

In terms of hardware, tests have been carried out on different object datasets with adversarial shapes and different gripper designs, e.g. RobotiQ 2F140 and RobotiQ 2F85 adaptive two-finger grippers, the suction foam FM-SW and the HGPC16A pneumatic two-finger gripper. The pipeline is designed to be easily expanded with new gripper models.

The proposal has produced viable and applicable grasping solutions, with a configured structure that allows the proposed system to be adapted to various UE R\&D projects: Fasten, Mary4Yard and Produtech 4S\&C. It also leads to the development of a mimic system with suitable hardware and a standard grasping dataset.

In addition, the current work developed a comprehensive study of the state*of-the-art in grasping, guiding to structured and standardised outcome assessments. These categories were published in 2021 and are already being used by other authors~\cite{hu2022grasps}. This shows that the scientific field lacks standardisation.

\section{Future Work}

The ``Grasping Synthesis Pipeline'' incorporates 3D model interaction, mathematical heuristics, optimisation techniques, and mimic grasping to create the candidate dataset. An improvement could be the addition of a AI and a machine learning branch, as this methods category have the advantage of a greater adaptability and generalisation. The deployment of 3D \ac{CNN}, such as Voxnet~\cite{maturana2015voxnet}, or multidimensional techniques, like the Multi-View Convolutional Neural Networks (MV -CNNs) proposed by~\cite{su2015multi} and used by~\cite{Mahler2016}, could be an option. These CNN architecture categories can identify features in a 3D database, such as a point cloud representation, and not under-valuate the database as an image classification problem. Works like~\cite{choi2018learning} demonstrate this capability. Although image-based \ac{DL} policies achieve interesting results in a \ac{GPSR} fashion, the \ac{GHSR} usually decreases because the image does not represent some critical parameters in grasping, e.g. three-dimensional representation and physical interaction. On the other hand, the creation of a large labeled 3D grasping database could require a significant effort. Another possible solution is to investigate the transfer learning techniques involving the already consolidated two-dimensional \ac{DL} policies to the new multidimensional \ac{DL} category. For this, the development of frameworks for automatic registration of grasping attempts by application robots (multi-agents) that share their knowledge in a cloud computing or IoT method is an interesting improvement in the pipeline. The developed dataset standard is a first step in this direction, but the ``Post-Processor'' pipeline needs to be improved with agnostic gripper design heuristics, e.g., using the grippers URDF models is possible to define the inverse kinematics with respect to the robot \acp{TCP}, which allows the correction grasping translation between different robot configurations. 

With respect to the proposed ``Mimic Grasping'' setup, the 6D Mimic design could be improved with a third perpendicular camera, since the precision of the stereo system in the baseline normal direction is compromised.

Finally, regarding the ``Grasping Selection'', a study is needed to define the use of a heuristic based on the placement of the object. Indeed, the placement of an object could be an important factor in the selection of candidates, i.e., some grasping postures could lead to impractical placement configurations.



  








