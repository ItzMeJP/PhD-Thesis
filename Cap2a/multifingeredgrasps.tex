\section{Multi-Fingered Grasping}
\label{sec:multifingered_grasping}

A multi-fingered grasping is realised over a set of contacts between the active pairs (the workpiece and the gripper). Therefore, the determination of a suitable configuration of independent grasping points is the primary step of the fingered grasping planning. 

The wrench vectors describe the forces and moments that influence a rigid body's dynamic. These vectors can be used to formulate grasping locations, and a wrench vector is presented below:  

\begin{equation}
\mathbf{w_{c}}=\left[\begin{array}{l}
\mathbf{f} \\
\boldsymbol{\tau}
\end{array}\right]
\end{equation}

\noindent
where $\mathbf{f}$ and $\boldsymbol{\tau}$ are the vector representations of the forces and the moments. The wrench vectors have 3 and 6 \acp{DOF} in the case of $\mathbb{R}^2$ and $\mathbb{R}^3$, respectively.

The contact models can be categorised as friction-less contact, friction contact (also named hard finger contact), and soft contact~\cite{murray1994mathematical}. The focus of this chapter will be the friction contact, since this model covers the main application field of this thesis.

%The friction contact model considers the mechanical interaction between the active pairs. Therefore, the wrench convex depends on the friction contact forces, described by Coulomb model of friction: Considering the normal force $\mathbf{f_n}$, and the tangential force $\mathbf{f_t}$, static friction occurs when there is no slipping between the two surfaces of contact, that is when $\left|\mathbf{f_{t}}\right| \leq \mu_{t} |\mathbf{f_{n}}|
%$ where $\mu_{t} $ is a positive value representing the static tangential coefficient of friction. Figure~\ref{fig:friction_contact} shows an example of hard finger contact, the geometric representation of the Coulomb’s law and the friction cone convex also defined as ${FC}_{c_i}$. 

The friction contact model considers the mechanical interaction between the active pairs which is defined by Coulomb model of friction: Considering the normal force $\mathbf{f_n}$, and the tangential force $\mathbf{f_t}$, static friction occurs when there is no slipping between the two surfaces of contact, that is when $\left|\mathbf{f_{t}}\right| \leq \mu_{t} |\mathbf{f_{n}}|
$ where $\mu_{t} $ is a positive value representing the static tangential coefficient of friction. Figure~\ref{fig:friction_contact} shows an example of hard finger contact, the geometric representation of the Coulomb’s law and the friction cone convex also defined as ${FC}_{c_i}$. 

\begin{figure}[h!]
\resizebox{0.75\textwidth}{!}{%
\begin{tcolorbox}
\centerline{\includegraphics[trim={0cm 0cm 0cm 0cm},clip,width=1\linewidth,angle=0]{Apendices/Figuras/friction_contact_gray_bg2.pdf}}
\end{tcolorbox}
\caption{Friction contact model and the geometric representation of Coulomb’s law (figure based on~\cite{murray1994mathematical}).}
\label{fig:friction_contact}
} %end resize box
\end{figure}

A wrench representation, w.r.t the $i$-th contact point ($c_i$), could be defined as follows:

\begin{equation}
\mathbf{W_{c_i}}=\begin{bmatrix}
1 & 0 & 0 \\ 
0 & 1 & 0 \\ 
0 & 0 & 1 \\ 
0 & 0 & 0 \\ 
0 & 0 & 0 \\ 
0 & 0 & 0 
\end{bmatrix} \mathbf{f}_{c_i}, \quad \mathbf{f}_{c_i} \in F C_{c_i}
\end{equation}

\noindent
where $F C_{c_i}=\mathbf{f} \in \!R^3: \sqrt{f_{x}^{2}+f_{y}^{2}} \leq \mu_{t} f_{z,}, f_{z} \geq 0$, and $\mu_t$ is the transversal friction coefficient in $c_i$. 

Therefore, based on this wrench representation, it is possible to define the matrix that compose the wrench vector concept:

\begin{equation}
\mathbf{W}_{c_i}=\mathbf{B}_{c_i} \mathbf{f}_{c_i}, \quad \mathbf{f}_{c_i} \in F C_{c_i}
\end{equation}

\noindent
where $\mathbf{{B}_{c_i}}$ is the wrench basis matrix with dimension $p \times n$ where $p$ is the \acp{DOF} and $n$ the number of independent forces and moments that constitutes $\mathbf{{f}_{c_i}}$. The contact model discussed here has as reference frame the one with the origin coincident with the contact point itself. It is more convenient to refer all contacts in a grasp model to a common frame, generally the centre of mass of the work piece. Therefore the wrench transformation matrix is defined as follows:

\begin{equation}
^{o} \mathbf{Tw}_{c_i}=\left[\begin{array}{cc}
{^{o} \mathbf{R}_{\mathrm{c_i}}} & {0} \\
{^o \hat{\mathbf{t}}_{\mathrm{c_i}} {}^{o} \mathbf{R}_{\mathrm{c_i}}} & {^{o} \mathbf{R}_{\mathrm{c_i}}}
\end{array}\right]
\in \mathbb{R}^3
\end{equation}

\noindent
where $ ^o\mathbf{R}_{c_i}$ and $^o\mathbf{t}_{c_i}$ are the rotation and translation matrix of the $i$-th contact point ($c_i$) w.r.t. object frame ($o$). The $\mathbf{\hat{t}}$ is the linear operator representing the cross product $^o\mathbf{t}_{c_i} \times {{{}^o}\mathbf{R}}_{c_i}$ as bellow:

\begin{equation}
\widehat{\boldsymbol{a}}=\left[\begin{array}{ccc}
0 & -a_{3} & a_{2} \\
a_{3} & 0 & -a_{1} \\
-a_{2} & a_{1} & 0
\end{array}\right] \quad where \quad \boldsymbol{a} = \left[\begin{array}{c}
a_{1} \\
a_{2} \\
a_{3}
\end{array}\right]
\end{equation}


Hence, the contact map $\mathbf{G}_{i}$ is defined as follows:

\begin{equation}
\mathbf{G}_{i}=^{o} \mathbf{T} \mathbf{w}_{c_i}^{\prime} \mathbf{B}_{c_i}
\end{equation}

Note that it describes the direction of each component of the $i$-th applied wrench and defines the constraints of the contact. The grasp map is the matrix with all contact maps that characterise the contact model (it is also named constraint matrix):

\begin{equation}
\mathbf{G}=\left[\begin{array}{llll}
{{}^o \mathbf{T} \mathbf{w}_{c_1}^{\prime} \mathbf{B}_{c_1}} & {\dots} & {{}^{o} \mathbf{T} \mathbf{w}_{c_N}^{\prime} \mathbf{B}_{c_N}}
\end{array}\right]
\end{equation}

Then, including the magnitude of the forces, a workpiece wrench can be written:

\begin{equation}
^{o} \mathbf{W}=\left[\mathbf{G}_{1}, \ldots, \mathbf{G}_{N}\right]\left[\mathbf{f}_{c_1}, \ldots, \mathbf{f}_{c_N}\right]^{\prime}=\mathbf{G} \mathbf{F}
\end{equation}

\noindent
where: $\mathbf{F} \in F C \text { and } F C=F C_{c_1} \times \ldots \times F C_{C_N}$

The grasp map is an important matrix since it is the mathematical formulation of a multi-finger grasping. From now on, the common reference frame is defined as the object frame, and for convenience, the $^{o}\mathbf{W}$ will be substituted by $\mathbf{W}$. Each column of $\mathbf{W}$ represents the independent contact wrenches. All the contacts discussed here are punctual contacts since the other kinds of contact can be approximate by a set of punctual contacts and edge contacts.

By means of the convex linearisation of the friction contact, Figure~\ref{fig:friction_contact_linearised}, it is possible to represent the contact force ($\mathbf{f}_{c_i}$) as a linear combination of the cone edges ($\mathbf{S}_{c_i}$) and ($a_d$):

\begin{figure}[h!]
	\resizebox{0.75\textwidth}{!}{%
		\begin{tcolorbox}
			\centerline{\includegraphics[trim={0cm 0cm 0cm 0cm},clip,width=1\linewidth,angle=0]{Apendices/Figuras/friction_contact_linearised.pdf}}
		\end{tcolorbox}
		\caption{Linearised friction contact model.}
		\label{fig:friction_contact_linearised}
	} %end resize box
\end{figure}


\begin{equation}
	\mathbf{f}_{ci} \approx \sum_{d=0}^{D} a_{d} \mathbf{s}_{ci_d}, a_{d} \geq 0 \quad, \quad \sum_{d=0}^{D} a_{d}=1
\end{equation}

where D is the number of edges that composes the linearised cone. A matrix notation can be formed as:

\begin{equation}
	\mathbf{f}_{ci}=\mathbf{A} \mathbf{S}_{ci}^{\prime}
\end{equation}

with $\mathbf{A}=\left[a_{0}, \ldots, a_{D}\right]$ and $\mathbf{S}_{ci}=\left[\mathbf{s}_{ci_0}, \ldots, \mathbf{s}_{ci_D}\right]$

Therefore, including all contact forces, the $\mathbf{W}$ associate with $\mathbf{S}$ is called primitive grasp map, also referred as primitive wrench map, defined as follow:

\begin{equation}
	\mathbf{W}_{p}=\left[\mathbf{G}_{1}, \ldots, \mathbf{G}_{N}\right]\left[\mathbf{A} \mathbf{S}_{c 1}^{\prime}, \ldots, \mathbf{A} \mathbf{S}_{c N}^{\prime}\right]^{\prime}=\mathbf{G} \mathbf{F}_{\mathbf{p}}
\end{equation}

The columns of $\mathbf{W}_{p}$ represent all wrenches associated with the edges of the linearised friction cone of the N contacts. It is an interesting matrix since it defines boundary conditions of contact. Applying a torque factor $\alpha$ (ensuring $|\tau| \leq|F|$) in each $w_{pd_{ci}}$ in $\mathbf{W}_{p}$, i.e.:

\begin{equation}
	w_{c}=\left[\begin{array}{c}
		\boldsymbol{f} \\
		\alpha \boldsymbol{\tau}
	\end{array}\right]
\end{equation}

and defining $\alpha=\frac{1}{r}$, where $r$ is the maximum radius of the centre of mass or gravity of the object and a contact point, it is possible to evaluate the grasps independently of the object size.

The $\mathbf{W}_p$ of all contact forces also defines the GWS (grasp wrench space) of the grasping. It is obtained by means of the $L_\infty$ or $L_1$ norm over the vector $\mathbf{g}$ composed by the magnitude of each contact normal force. The $L_\infty$ defines the GWS($W_L\infty$) considering the limitation of the maximum allowable normal contact force, while $L_1$ defines the GWS($W_{L1}$) by the sum magnitude of the normal contact forces. The norms operation yields to:


\begin{equation}
\begin{aligned}
\mathbf{W}_{L_{1}} &=\text { ConvexHull }\left(\bigcup_{c_i}^{N} \mathbf{w}_{p_1 c_i}, \ldots, \mathbf{w}_{p D_{c_i}}\right) \\
\mathbf{W}_{L_{\infty}} &=\text { ConvexHull }\left(\bigoplus_{c i}\left\{\mathbf{w}_{p_1 c_i}, \ldots, \mathbf{w}_{p D_{c_i}}\right\}\right)
\end{aligned}
\end{equation}

\noindent
where $\mathbf{w}_{pd_{ci}}$ $\in$ $\mathbf{W}$ and $\bigoplus$ is the Minkowski sum. More detail about the norm operation can be verified in~\cite{Ferrari}.


The concept of grasp closure evaluates the restraining of an object. A common assumption is the force-closure implies an equilibrium, but the inverse does not apply. A grasp has its convex hull defined by the wrenches that constitute the grasp configuration, i.e., the matrix $\mathbf{W}$. In a force-closure grasp, the convex hull includes the wrench space origin $\{O\}$, see Figure~\ref{fig:gws_force_closure}. According to the definition presented in~\cite{salisbury1983kinematic}, if all wrenches in $\mathbf{W}$ positively span the entire wrench space, the grasp will be force-closure. Figure~\ref{fig:gws_force_closure} shows a  grasp wrench space (GWS) and a convex hull of grasp configuration for force and non-force-closure, for a planar case with a fixed value for the moment ($\tau$) in the z-axis. Therefore, it is considered $\mathbf{f}_{c_i} \in  \mathbb{R}^2:  {f}_{c_i}  = (f_{x} , f_{y})$, and  the resistance to perturbation in both force axes is evaluated. 

\begin{figure}[h!]
\resizebox{0.8\textwidth}{!}{%
\begin{tcolorbox}
\centerline{\includegraphics[trim={0cm 1.2cm 0cm 5.5cm},clip,width=1\linewidth,angle=0]{Cap2/Figuras/wrenchspace_anlayses_2.pdf}}
\end{tcolorbox}
\caption{Wrench convex-hull configuration. Force-closure and the $\epsilon$-value (left). Non-force-closure (right).}
\label{fig:gws_force_closure}
}%end resize box
\end{figure}

Since several configurations can reach a force-closure grasp, quality metrics like $\epsilon$-metric evaluate which one is best. The $\epsilon$ is a normalized value that represents the wrench vector's distance to the origin ($\{O\}$), which is the shortest, i.e., the worst wrench vector to support an external perturbation. An efficient grasp, ideally, has $\epsilon=1$ . The left GWS of Figure~\ref{fig:gws_force_closure} elucidates this metric and the readers are encouraged to a more detailed review of this grasping definition in~\cite{Ferrari}.


